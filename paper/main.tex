%%
%% This is file `sample-sigconf.tex',
%% generated with the docstrip utility.
%%
%% The original source files were:
%%
%% samples.dtx  (with options: `sigconf')
%% 
%% IMPORTANT NOTICE:
%% 
%% For the copyright see the source file.
%% 
%% Any modified versions of this file must be renamed
%% with new filenames distinct from sample-sigconf.tex.
%% 
%% For distribution of the original source see the terms
%% for copying and modification in the file samples.dtx.
%% 
%% This generated file may be distributed as long as the
%% original source files, as listed above, are part of the
%% same distribution. (The sources need not necessarily be
%% in the same archive or directory.)
%%
%% The first command in your LaTeX source must be the \documentclass command.
\documentclass[sigconf,natbib=false]{acmart}

\usepackage[backend=biber]{biblatex}
\addbibresource{ref.bib}

%%
%% \BibTeX command to typeset BibTeX logo in the docs

%% Rights management information.  This information is sent to you
%% when you complete the rights form.  These commands have SAMPLE
%% values in them; it is your responsibility as an author to replace
%% the commands and values with those provided to you when you
%% complete the rights form.
\setcopyright{acmcopyright}
\copyrightyear{2018}
\acmYear{2018}
\acmDOI{10.1145/1122445.1122456}


%% end of the preamble, start of the body of the document source.
\begin{document}

\title{Network storage with P4}

%%
%% The "author" command and its associated commands are used to define
%% the authors and their affiliations.
%% Of note is the shared affiliation of the first two authors, and the
%% "authornote" and "authornotemark" commands
%% used to denote shared contribution to the research.
\author{Tamas Lengyel}
\email{d1b5hi@inf.elte.hu}

\author{Noel Hetei}
\email{njoim8@inf.elte.hu}

\author{Peter Kokai}
\email{kokaipeter@gmail.com}


%%
%% The abstract is a short summary of the work to be presented in the
%% article.
\begin{abstract}
Using P4, python and MiniNet our task was to create a network storage. With a contoller data should could be uploaded, queried and removed from the topology. Using a pre-made project as a starting point, we could implement a working but simplified solution. During implementation we have encountered numerous issues from storing the data to cloning the data to give proper response to the user. 
\end{abstract}

%%
%% The code below is generated by the tool at http://dl.acm.org/ccs.cfm.
%% Please copy and paste the code instead of the example below.
%%
\begin{CCSXML}
<ccs2012>
<concept>
<concept_id>10003033.10003099.10003102</concept_id>
<concept_desc>Networks~Programmable networks</concept_desc>
<concept_significance>500</concept_significance>
</concept>
<concept>
<concept_id>10003033.10003099.10003103</concept_id>
<concept_desc>Networks~In-network processing</concept_desc>
<concept_significance>300</concept_significance>
</concept>
<concept>
<concept_id>10003033.10003039.10011655</concept_id>
<concept_desc>Networks~Network File System (NFS) protocol</concept_desc>
<concept_significance>100</concept_significance>
</concept>
</ccs2012>
\end{CCSXML}

\ccsdesc[500]{Networks~Programmable networks}
\ccsdesc[300]{Networks~In-network processing}
\ccsdesc[100]{Networks~Network File System (NFS) protocol}

%%
%% Keywords. The author(s) should pick words that accurately describe
%% the work being presented. Separate the keywords with commas.
\keywords{p4lang, network storage,}

%%
%% This command processes the author and affiliation and title
%% information and builds the first part of the formatted document.
\maketitle

\section{Introduction}
P4 is a language to program network switches. It defines how a switch proccesses incoming and outgoing packets. Using python scripting language and MiniNet, which creates a virtual network, we are going to create a network storage where packets circulating from switch to switch. The main goal is with a controller, written in python, we are able to upload, query and remove data from a custom topology. Swicthes are programmed in P4. Furthermore, certain methods are going to be described in details not to mention some obsticles with simplifications to achive the expected goal.

\section{Overveiw}
In this section we are going show how the project evolved step by step until it reached its final form. From the ``template" we used to start to explain how it works and how to use it.
\subsection{Preparation}
Setting up a P4 project is not an easy task because of the lack of documentation of the language. Other reason is that most of us has not worked with P4 before so we decided to go with a pre-made project and modify that for our purposes. We used the ``cacl" exercise project from the \textbf{p4lang/tutorials} - \url{https://github.com/p4lang/tutorials} - where a premade virutal machine is defined as well. Running the \textit{vagrant up} command starts the virtual machine - from now on we use the short form: \textit{vm} -  with all of its dependencies like MiniNet or Python. The process pipline has not been touched for the switches.
\subsection{Tasks}
We had several features to implement with question that had been raised.
\begin{itemize}
	\item {\verb|Storing data|}: Where and how to store the incoming data?
	\item{\verb|Implementing operation|}: How to handle incoming operations?
	\item{\verb|Handling Parallel Requests|}: What happens when multiple clients connect?
	\item{\verb|Multiple Servers|}: How to identify the data?
	\item{\verb|Handling Large Data|}: What happens when large amount of data arrive to a switch?
\end{itemize}
Some questions have simple answers. For example operation implementation where the template contained exapmles codes of it. It was easy to program them on the client side and for the switch to recognise each operation\footnote{Get and put operations have cloning which made them hard to code and test which is going to be discussed later}. Some items had to be simplified to make them work.

\section{Challenges \& Simplifications}
Each item has its own challenge. Considering the limited time we had some parts of the tasks had to be simplified in order to finish the project on time and not the sit on a feature implementation  for weeks or even months.
\subsection{Storing Data}
Storing data has raised to question that had to be considered. \textit{What kind of data structure should we use?} anfd \textit{How the servers positioned in the topology?} The solution for the first issue was found after some searching on the web that the header is capable of stroing data, which was enough for us. We decided to simplify the second question by using only simple topologies where each switch has two or maximum three connections. The data can circulate inside the topology using any port.
\subsection{Implementing Operations}
The ``template" project we used had four precoded operation which could be used as a basis for our own implementations. The put and query operations has cloning implemented to them in order to send responses to the client. It was one of the biggest challenges that the team encountered. Minor simplification here is that we use port number three for communicating with the client. It is necessary because of the cloning procedure which uses a port that has been hard coded into the system.
\subsection{Handling Parallel Request}
Handling multiple request at the same time is hard to implement. One possible solution is to use working queues. Due to time limitations we had to simplify the task so we have assumed that only one user sends data from a single computer into the topology.
\subsection{Multiple Servers}
In order to maintain consistency a unique id had to be assigned for each datum. First, we tried to use a simple counter. But when there are multiple servers this system can break easily so we decided to use hashing.
\subsection{Handling Large Data}
Large data can be handled server or client side. First, we wanted to keep ourself away from splitting the data into multiple packets. So we ahve agreed to impelent a data size restriction only on the client side. If the users tries to send a data that is larger than the defiened value the datum will not be sent into the topology.

\section{Methods}

The title of your work should use capital letters appropriately -
\url{https://capitalizemytitle.com/} has useful rules for
capitalization. Use the {\verb|title|} command to define the title of
your work. If your work has a subtitle, define it with the
{\verb|subtitle|} command.  Do not insert line breaks in your title.

If your title is lengthy, you must define a short version to be used
in the page headers, to prevent overlapping text. The \verb|title|
command has a ``short title'' parameter:
\begin{verbatim}
  \title[short title]{full title}
\end{verbatim}

\section{Tests \& Results}

Each author must be defined separately for accurate metadata
identification. Multiple authors may share one affiliation. Authors'
names should not be abbreviated; use full first names wherever
possible. Include authors' e-mail addresses whenever possible.

Grouping authors' names or e-mail addresses, or providing an ``e-mail
alias,'' as shown below, is not acceptable:
\begin{verbatim}
  \author{Brooke Aster, David Mehldau}
  \email{dave,judy,steve@university.edu}
  \email{firstname.lastname@phillips.org}
\end{verbatim}

The \verb|authornote| and \verb|authornotemark| commands allow a note
to apply to multiple authors --- for example, if the first two authors
of an article contributed equally to the work.

If your author list is lengthy, you must define a shortened version of
the list of authors to be used in the page headers, to prevent
overlapping text. The following command should be placed just after
the last \verb|\author{}| definition:
\begin{verbatim}
  \renewcommand{\shortauthors}{McCartney, et al.}
\end{verbatim}
Omitting this command will force the use of a concatenated list of all
of the authors' names, which may result in overlapping text in the
page headers.

The article template's documentation, available at
\url{https://www.acm.org/publications/proceedings-template}, has a
complete explanation of these commands and tips for their effective
use.

Note that authors' addresses are mandatory for journal articles.

\section{Conclusion}

Authors of any work published by ACM will need to complete a rights
form. Depending on the kind of work, and the rights management choice
made by the author, this may be copyright transfer, permission,
license, or an OA (open access) agreement.

Regardless of the rights management choice, the author will receive a
copy of the completed rights form once it has been submitted. This
form contains \LaTeX\ commands that must be copied into the source
document. When the document source is compiled, these commands and
their parameters add formatted text to several areas of the final
document:
\begin{itemize}
\item the ``ACM Reference Format'' text on the first page.
\item the ``rights management'' text on the first page.
\item the conference information in the page header(s).
\end{itemize}

Rights information is unique to the work; if you are preparing several
works for an event, make sure to use the correct set of commands with
each of the works.

The ACM Reference Format text is required for all articles over one
page in length, and is optional for one-page articles (abstracts).

\section{CCS Concepts and User-Defined Keywords}

Two elements of the ``acmart'' document class provide powerful
taxonomic tools for you to help readers find your work in an online
search.

The ACM Computing Classification System ---
\url{https://www.acm.org/publications/class-2012} --- is a set of
classifiers and concepts that describe the computing
discipline. Authors can select entries from this classification
system, via \url{https://dl.acm.org/ccs/ccs.cfm}, and generate the
commands to be included in the \LaTeX\ source.

User-defined keywords are a comma-separated list of words and phrases
of the authors' choosing, providing a more flexible way of describing
the research being presented.

CCS concepts and user-defined keywords are required for for all
articles over two pages in length, and are optional for one- and
two-page articles (or abstracts)\cite{igarashi2005lightweight}.

\section{Sectioning Commands}

Your work should use standard \LaTeX\ sectioning commands:
\verb|section|, \verb|subsection|, \verb|subsubsection|, and
\verb|paragraph|. They should be numbered; do not remove the numbering
from the commands.

Simulating a sectioning command by setting the first word or words of
a paragraph in boldface or italicized text is {\bfseries not allowed.}

\begin{equation}
  \lim_{n\rightarrow \infty}x=0
\end{equation}

\begin{displaymath}
  \sum_{i=0}^{\infty} x + 1
\end{displaymath}
and follow it with another numbered equation:
\begin{equation}
  \sum_{i=0}^{\infty}x_i=\int_{0}^{\pi+2} f
\end{equation}


\section{Citations and Bibliographies}


\section{Acknowledgments}

%%
%% The next two lines define the bibliography style to be used, and
%% the bibliography file.
\printbibliography



\end{document}
\endinput
%%
%% End of file `sample-sigconf.tex'.
